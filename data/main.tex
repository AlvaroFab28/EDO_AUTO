% Archivo: Banco/estandar.tex
\documentclass{article}
\usepackage[utf8]{inputenc}
\usepackage{amsmath,amssymb}
\usepackage{graphicx}

\title{Banco de Ejercicios EDO}
\author{Molina Yampa}
\date{Septiembre 2025}

\begin{document}
\maketitle

% -------------------------
% Formato estándar de parser:
% - Cada ejercicio va entre:
%     %% EXERCISE_START
%     ... (un \[ ... \] con el ENUNCIADO)
%     ... (un \[ ... \] con la RESPUESTA)
%     %% EXERCISE_END
% - Si la condición existe, debe escribirse inline al final del primer \[ ... \]
%   utilizando la secuencia: ", \quad <condición>" (ej.: ", \quad y(0)=1").
% - El parser tomará la última \section / \section* previa como "sección" y la última
%   \subsection / \subsection* previa como "subsección".
% -------------------------

\section{Ecuaciones Diferenciales de Variable Separable}

%% EXERCISE_START
% id: 1
1)
\[
\tan x \cdot \sin^2 y \, dx + \cos^2 x \cdot \cot y \, dy = 0, \quad y(1) = 2
\]
\[
c\,\tan^2 y = \tan^2 x + C
\]
%% EXERCISE_END

\vspace{8pt}

%% EXERCISE_START
% id: 2
2)
\[
xy' - y = y^3
\]
\[
x = \frac{c\,y}{\sqrt{1 + y^2}}
\]
%% EXERCISE_END

\vspace{8pt}

%% EXERCISE_START
% id: 3
3)
\[
\sqrt{1 + x^3}\,\frac{dy}{dx} = x^2 y + x^2
\]
\[
2\sqrt{1 + x^3} = 3\ln(y + 1) + C
\]
%% EXERCISE_END

\vspace{8pt}

%% EXERCISE_START
% id: 4
4)
\[
\sin 2x \, dx + \cos 3y \, dy = 0, \quad y\!\left(\frac{\pi}{2}\right) = \frac{\pi}{3}
\]
\[
2 \sin 3y - 3 \cos 2x = 3
\]
%% EXERCISE_END

\vspace{8pt}

%% EXERCISE_START
% id: 5
5)
\[
y' - 2y \cot x = 0, \quad y\!\left(\frac{\pi}{2}\right) = 2
\]
\[
y = 2 \sin x
\]
%% EXERCISE_END

% -------------------------
% Nuevo subtema: solución particular con condiciones
% (sección visible; el parser tomará todo lo que venga hasta nueva sección)
% -------------------------
\section*{II. Encontrar la solución particular de la ecuación diferencial, mediante las condiciones dadas.}

\vspace{6pt}

%% EXERCISE_START
% id: II-1
1)
\[
\frac{dy}{dx} = \frac{x - y}{1 + y'}, \quad y(0) = 1
\]
\[
3y^2 + 2y^3 = 3x^2 + 5
\]
%% EXERCISE_END

\vspace{8pt}

%% EXERCISE_START
% id: II-2
2)
\[
x(y^6 + 1)\,dx + y^2(x^4 + 1)\,dy = 0, \quad y(0) = 1
\]
\[
3 \arctan x^2 + 2 \arctan y^3 = \frac{\pi}{2}
\]
%% EXERCISE_END

\vspace{8pt}

%% EXERCISE_START
% id: II-3
3)
\[
y' \sin x = y \ln y,\quad y\!\left(\frac{\pi}{2}\right) = e
\]
\[
y = e^{\frac{x}{2}}
\]
%% EXERCISE_END

\vspace{10pt}

% Puedes añadir más ejercicios repitiendo exactamente este patrón.
% El parser no necesita ver "Enunciado/Respuesta/Condición" en texto:
% usa los comentarios %% EXERCISE_START / %% EXERCISE_END y los bloques \[...\] para encontrar los campos.
% Las condiciones deben ir inline dentro del primer \[...\] con ", \quad <condición>".

\end{document}
